\chapter{Use Case: Flights and Airports}

In this module we work on the main use-case of the class.  This use-case entails combining flights with airports and transforming the data such that are communicated with external systems. The tasks we shall perform are as follows:
\begin{itemize}
\item Dynamically rename fields
\item Combine data from two sources
\item Explore in more detail functional programming
\item Optimize your code in the absence of profiling tools
\item Reorder objects to meet legacy system criteria
\item Traverse and transform any type of structured data
\end{itemize}
In fact, the remainder of the class makes use of this use-case.  We shall still apply unit-testing to solve smaller issues before we integrate the unit-tested expressions in our main transformation. 

%%
%%
%%
\section{Change field names}

%%%%%%
\subsection{Create a new flow}
\begin{enumerate}
\item Create a new Mule Configuration file and name it \ttt{mod4}
\item Create a new flow named \ttt{mod4-flights-airports}
\item Drop a DW to the process area of the flow
\item Define the payload input metadata to  \ttt{flights\_json}
\item Edit the sample data
\item Turn on the preview
\item Change the output to \ttt{application/dw}
\end{enumerate}

%%%%%%
\subsection{Create the map}
\begin{enumerate}[resume*]
\item Create a variable, \ttt{fs2fs}, that contains a map from source field names to target field names
  \begin{verbatim}
    var fs2rn = {
      airlineName: "carrier",
      departureDate: "date",
      emptySeats: "seats",
      planeType: "plane"
    }
  \end{verbatim}
  \emph{
    I do know we can just do it in-place as part of the lambda expression of a \ttt{map} function. Creating the variable will allow to show how it can be done dynamically.  I could easily pass such a map through an HTTP request, read it from a file, etc.
  }
  \newline
  \emph{
    What we need to do is iterate over an object!
  }
\end{enumerate}

%%%%%%
\subsection{The \ttt{mapObject} function}
\begin{enumerate}[resume*]
\item Access the first object
  \begin{verbatim}
    %dw 2.0
    output application/dw
    
    var fs2rn = {
      airlineName: "carrier",
      departureDate: "date",
      emptySeats: "seats",
      planeType: "plane"
    }
    ---
    payload[0]
  \end{verbatim}
\item Visit the \href{https://docs.mulesoft.com/mule-runtime/4.3/dw-core-functions-mapobject}{\ttt{mapObject} documentation page} and explore its signature.  Focus on the return type of the \les{}.
\item Iterate over the object and return the empty object form the \les
  \begin{verbatim}
    payload[0] mapObject (v,k,i) -> {}
  \end{verbatim}
  \emph{
    Notice the result in the preview, it is also an empty object.  \ttt{mapObject} invokes the \les{} once for every single pair of key and value in order of appearance.  The result of \ttt{mapObject} is the concatenated objects returned from the \les.
  }
\item Reform the object
  \begin{verbatim}
    payload[0] mapObject (v,k,i) -> {k: v}
  \end{verbatim}
\item Examine the preview
  \begin{verbatim}
    {
      k: "Delta",
      k: "A1B2C3",
      k: "2018/03/20",
      k: "SFO",
      k: "40",
      k: "MUA",
      k: "Boing 737",
      k: "400.0"
    }
  \end{verbatim}
  \emph{
    Notice that the values are being reformed, that is the \ttt{v} is evaluated, while the keys, \ttt{k}, is displayed verbatim.
  }
\item Force the evaluation of the field name
  \begin{verbatim}
    payload[0] mapObject (v,k,i) -> {(k): v}
  \end{verbatim}
  \emph{
    We must enclose the key inside \ttt{()} to force its evaluation.  It is how the language works to facilitate the creation of fields--on the one hand we don't have to make excessive use of quotes enclosing the hard-coded fields while on the other we must enclose the keys inside parenthesis when dynamically evaluating them.
  }
  \newline
  \emph{
    It is worth the effort to inspect the types of fields when using the \ttt{mapObject} function.  By enclosing the \ttt{(sizeOf(k))} you will see in the preview that fields are of a specific type \ttt{Key}.  This is a critical detail that we will latch on when working on the final exercise of the class. 
  }
\end{enumerate}

%%%%%%
\subsection{Change the field names}
\begin{enumerate}[resume*]
\item Dynamically evaluate the key based upon the \ttt{fs2rn} map for a single object
  \begin{verbatim}
    payload[0] mapObject (v,k,i) -> {(fs2rn[k]): v}
  \end{verbatim}
\item Examine the error: \ttt{Cannot coerce Null (null) to Key}
  \newline
  \emph{
    The error is due to keys not appearing in the \ttt{fs2rn} map.  For example, \ttt{code}, \ttt{price}, etc.
  }
\item Accommodate for unchanged keys
  \begin{verbatim}
    payload[0] mapObject (v,k,i) -> {(fs2rn[k] default k): v}
  \end{verbatim}
\item Change the field names for all objects in the array
  \begin{verbatim}
    payload map (
      $ mapObject (v,k,i) -> {(fs2rn[k] default k): v}
    )
  \end{verbatim}
\end{enumerate}

%%
%%
%%
\section{Combine flights and airports}

%%%%%%
\subsection{Explore the CSV file}
\begin{enumerate}
\item Navigate under \ttt{src/main/resources} folder
\item Right click the \ttt{airport-info.csv}
\item Select \ttt{Open With} and then click \ttt{Text Editor}
\item Discuss the contents and identify bad data
  \newline
  \emph{
    Dealing with bad data now days is the norm, as such this CSV file contains bad data.  There are two instances of such bad data:
    \begin{itemize}
    \item In the header the \ttt{ICAO} header name is prefixed with an empty space
    \item Rows 7 and 13 are duplicates; this CSV file should contain unique records
    \end{itemize}
  }
\end{enumerate}


%%%%%%
\subsection{Parse the CSV airports}
\begin{enumerate}[resume*]
\item Navigate back to \ttt{mod4} mule configuration file
\item Create a new flow named \ttt{mod4-read-parse-airports}
\item Drop a DW to the process area of the flow
\item Turn on the preview
\item Change the output to \ttt{application/dw}
\item Just type the function \ttt{readUrl} in the body of the expression
  \begin{verbatim}
    %dw 2.0
    output application/dw
    ---
    readUrl
  \end{verbatim}
\item Explore the signature
  \begin{verbatim}
    (first:String | Binary, second:String, third:Object) -> ???
  \end{verbatim}
  \emph{
    The details of \ttt{readUrl} function can be found in this \href{https://docs.mulesoft.com/mule-runtime/4.3/dw-core-functions-readurl}{page}.
  }
\item Read and parse the file
  \begin{verbatim}
    %dw 2.0
    output application/dw
    ---
    readUrl(
      "classpath://airport-info.csv",
      "application/csv",
      {
        separator: ","
      }
    )
  \end{verbatim}
  \emph{
    The last argument, aka reader properties, is not needed because the default separator is the \ttt{,} symbol.  It is only here for illustration.  You can find all reader properties, along with the default values, in this \href{https://docs.mulesoft.com/mule-runtime/4.3/dataweave-formats-csv\#reader-properties}{documentation page}.
  }
\item Fix the duplicate records
  \begin{verbatim}
    %dw 2.0
    output application/dw
    ---
    readUrl(
      "classpath://airport-info.csv",
      "application/csv",
      {
        separator: ","
      }
    )
    distinctBy $.IATA
  \end{verbatim}
  \emph{
    \ttt{IATA} stands for International Air Transport Association and its values are standardized across the industry to be unique codes identifying airports.  As such we make use of just this value to create a set out the collection--i.e. get the unique values.
  }
\item Incorporate the code in the \ttt{mod4-flights-airports} flow
  \begin{verbatim}
    %dw 2.0
    output application/dw

    var fs2rn = {
      airlineName: "carrier",
      departureDate: "date",
      emptySeats: "seats",
      planeType: "plane"
    }

    var airports = readUrl(
      "classpath://airport-info.csv",
      "application/csv",
      {
        separator: ","
      }
    )
    distinctBy $.IATA

    ---
    payload map (
      $ mapObject (v,k,i) -> {(fs2rn[k] default k): v}
    )
  \end{verbatim}
\end{enumerate}

%%%%%%
\subsection{Inject the airport to each flight}
\begin{enumerate}[resume*]
\item Chain another \ttt{map} function
  \begin{verbatim}
    payload map (
      $ mapObject (v,k,i) -> {(fs2rn[k] default k): v}
    )
    map {
      
    }
  \end{verbatim}
  \emph{
    There is no need for another chained \ttt{map} function, can can embed the combination of the data in a single iteration.  However, since we are in training clarity takes precedence vs performance.
  }
\item Reintroduce the fields from the previous map
  \begin{verbatim}
    map {
      ($)
    }
  \end{verbatim}
\item Add the \ttt{airport} field and assign to it all airports
  \begin{verbatim}
    map {
      ($),
      airport: airports
    }
  \end{verbatim}
\item Nest \ttt{filter} to search for the \ttt{SFO} airport
  \begin{verbatim}
    map {
      ($),
      airport: airports filter ($.IATA == "SFO")
    }
  \end{verbatim}
\item Dynamically make use of the flight's \ttt{destination} field to search the airports
  \begin{verbatim}
    map {
      ($),
      airport: airports filter ($.IATA == $.destination)
    }
  \end{verbatim}
\item Examine the output in the preview
  \begin{verbatim}
    [
      {
        carrier: "Delta",
        code: "A1B2C3",
        date: "2018/03/20",
        destination: "SFO",
        seats: "40",
        origin: "MUA",
        plane: "Boing 737",
        price: "400.0",
        airport: []
      },
      ...
    ]
  \end{verbatim}
  \emph{
    Notice the empty data when earlier we were displaying the SFO record.  The reason for this is because we have nested a filter right inside a map function and we choose to make use of the \ttt{\$} placeholders in both cases!
  }
\item Change \ttt{filter}'s \les{} argument names fixing the transformation
  \begin{verbatim}
    map {
      ($),
      airport: airports filter  (e) -> (e.IATA == $.destination)
    }
  \end{verbatim}
  \emph{
    Not only we fixed our transformation by combining each flight with an airport record we also fixed the airport record that is associated with a flight to match the flight's record \ttt{destination} field value.
  } 
\end{enumerate}

%%%%%%
\subsection{Functions as values}
\begin{enumerate}[resume*]
\item Create and apply a function that does the filtering
  \begin{verbatim}
    var filterAirportsByIATA = (dest) -> airports filter ($.IATA == dest)
    ---
    payload map (
      $ mapObject (v,k,i) -> {(fs2rn[k] default k): v}
    )
    map {
      ($),
      airport: filterAirportsByIATA($.destination)
    }
  \end{verbatim}
  \emph{
    For the most part the name of this functions gives away that every single aspect is hard-coded, apart from the IATA code passed to the \ttt{dest} argument.
  }
\item Create and apply a ``naive'' generic filter function
  \begin{verbatim}
    var filterAirportsByIATA = (dest) -> airports filter ($.IATA == dest)
    var genericFilter = (a,f,v) -> a filter ($[f] == v)
    ---
    payload map (
      $ mapObject (v,k,i) -> {(fs2rn[k] default k): v}
    )
    map {
      ($),
      airport: genericFilter(airports,"IATA",$.destination)
    }
  \end{verbatim}
  \emph{
    This is a naive version of a generic filter because we have hard-coded the condition!  A generic filter function should be able to take any kind of a \ttt{Boolean} condition and filter records of data.
  }
\item Rename the \ttt{genericFilter} function to \ttt{genericFilterNaive}
\item Create and apply a truly generic function
  \begin{verbatim}
    var filterAirportsByIATA = (dest) -> airports filter ($.IATA == dest)
    var genericFilterNaive = (a,f,v) -> a filter ($[f] == v)
    var genericFilter = (a,fn) -> a filter fn($)
    ---
    payload map (
      $ mapObject (v,k,i) -> {(fs2rn[k] default k): v}
    )
    map {
      ($),
      airport: genericFilter(airports,(airport) -> airport.IATA == $.destination)
    }
  \end{verbatim}
  \emph{
    Those observant enough should be able to realize the \ttt{filter} function is generic enough and there is no need for yet another function.  Nonetheless, this small exercise exposes us to aspects of functional programming and give us insights on how built-in functions that accept \lep{} as arguments behave.
    \newline
    This is an example where a function is passed as an argument to yet another function.  In a very similar manner we can return a function as the return value of a function--we shall see this next.
  }
\end{enumerate}

%%%%%%
\subsection{Optional: Curried functions}
\begin{enumerate}[resume*]
\item Create a curried function
  \begin{verbatim}
    var genericFilterC = (a) -> (fn) -> a filter fn($)
  \end{verbatim}
  \emph{
    Nope there is no relationship to the spice, the name comes from \href{https://en.wikipedia.org/wiki/Haskell_Curry}{Haskell Brooks Curry}.  Curry invented the concept of \href{https://en.wikipedia.org/wiki/Currying}{currying}
    \newline
    There are two main reasons for creating curried functions:
    \begin{itemize}
    \item Partial application of functions when all the arguments are not present at the same time--I have yet to find a use case where this partial application of curried functions can be applied in DW
    \item Function factories where out of a single definition you can define different but related functions.  This last point is where curried functions can apply in DW. 
    \end{itemize}
  }
\item Define functions for filtering flights and airports out of the \ttt{genericFilterC} function
  \begin{verbatim}
    var genericFilterC = (a) -> (fn) -> a filter fn($)
    var flightsFilter = genericFilterC(payload)
    var airportsFilter = genericFilterC(airports)
  \end{verbatim}
  \emph{
    Only one of the arguments is applied to \ttt{genericFilterC} and a function is returned and stored in the corresponding variables (i.e. \ttt{flightsFilter} and \ttt{airportsFilter}).
    \newline
    Another way to visualize these two function is by illustrating what they look like:
    \begin{itemize}
    \item flightsFilter: \ttt{(fn) -> payload filter fn(\$)}
    \item airportsFilter: \ttt{(fn) -> airports filter fn(\$)}
    \end{itemize}
    \href{https://stackoverflow.com/questions/36314/what-is-currying}{This discussion} under stackoverflow could also help with your understanding.
  }
\item Apply the \ttt{airportsFilter} function
  \begin{verbatim}
    map {
      ($),
      airport: airportsFilter((airport) -> airport.IATA == $.destination)
    }
  \end{verbatim}
\end{enumerate}

%%%%%%
\subsection{A more efficient transformation}
\begin{enumerate}[resume*]
\item Calculate the complexity of our algorithm
  \newline
  \emph{
    This is more of a discussion and a pen-and-paper exercise.  The essence of this discussion is a calculation, the Big-O notation.  Big-O notation is an abstract notation that measures the number of abstract operations that our algorithm will perform.  Our calculation will be a function of the number of flights and the number of airports.  Our focus is the iterations our algorithm performs, as such:
    \begin{itemize}
    \item Assume we have \ttt{N} number of flights
    \item Assume we have \ttt{M} number of airports
    \item \ttt{2*M} operations from the expression that reads, parses, and obtaining the set of airports
    \item \ttt{8*N} operations from the expression that renames dynamically the fields
    \item \ttt{N*M} operations from the combination of flights and airports
    \end{itemize}
  }
  \emph{
    Thus the total number of operations is calculated by the \ttt{2*M + 8*N + N*M} polynomial.  Remember the purpose of Big-O: abstractly identify the number of operations our algorithm performs to get the job done.  Splitting your Big-O notation using polynomials such as the one we just created identifies areas of your algorithm that are candidates for optimization.  
    \newline
    Replace \ttt{N} and \ttt{M} with a number of reasonable size (anything greater than 10 will do).  Which one of the operations in the polynomial will result in the largest number?  More importantly, if you keep calculating these operations with different number is the growth linear or exponential?  If the growth is exponential then these portions of your algorithm are prime candidates for optimization.  A linear growth is what you are after!
    \newline
    The growth of \ttt{N*M} is exponential or approaches an exponential growth.  When performing lookups ensure we have atomic lookups as compared searching arrays.  The ideal data structure to perform such atomic lookups is a map structure.
  }
\item Go to the DW processor of the \ttt{mod4-read-parse-airports} flow
\item Change the airports array into an object (aka map)
  \begin{verbatim}
    %dw 2.0
    output application/dw
    ---
    readUrl(
      "classpath://airport-info.csv",
      "application/csv",
      {
        separator: ","
      }
    )
    distinctBy $.IATA
    groupBy $.IATA
  \end{verbatim}
\item Explore the result in the preview
  \newline
  \emph{We can find the documentation page for \ttt{groupBy} \href{https://docs.mulesoft.com/mule-runtime/4.3/dw-core-functions-groupby}{here}.  In essence, \ttt{groupBy} takes the input array and transforms it into an object where the field names are set to the values the \les{} \ttt{\$.IATA} evaluates to.
  }
\item Apply the \ttt{groupBy} function to the \ttt{airports} variable in \ttt{mod4-flights-airports} flow
  \begin{verbatim}
    var airports = readUrl(
      "classpath://airport-info.csv",
      "application/csv",
      {
        separator: ","
      }
    )
    distinctBy $.IATA
    groupBy $.IATA
  \end{verbatim}
  \emph{
    We should be getting an error now.  The error identifies a type-mismatch on the \ttt{filter} function--i.e. An object is passed when an array is expected. 
  }
\item Change our expression to perform atomic lookups
  \begin{verbatim}
    payload map (
      $ mapObject (v,k,i) -> {(fs2rn[k] default k): v}
    )
    map {
      ($),
      airport: airports[$.destination]
    }
  \end{verbatim}
  \emph{
    A couple of notes to conclude this WT:
    \begin{itemize}
    \item The new complexity now is \ttt{3*M + 8*N + N}, we reached linear growth.
    \item The complexity we reached is related to processing time and we did not bother with the spatial complexity of our algorithm--i.e. the size of memory we use.  Perhaps this will be a good exercise for the students to complete on their own.
    \item Finally, we wonder why go through creation of a generic filter function to begin with?  The answer is we are getting exposure to functional programming where the knowledge thereof is crucial to all aspiring DW programmers.
    \end{itemize}
  }
\end{enumerate}

%% %%%%%%
%% \subsection{Optional: Clean up the data}
%% \begin{enumerate}[resume*]
%% \item Remove the \ttt{timeZone}, \ttt{type}, and \ttt{source} from the \ttt{airport} object
%%   \begin{verbatim}
%%     map {
%%       ($),
%%       airport: airports[$.destination][0] - "timeZone" - "type" - "source"
%%     }
%%   \end{verbatim}
%% \item Remove the \ttt{destination} field from each flight
%%   \begin{verbatim}
%%     map {
%%       ($ - "destination"),
%%       airport: airports[$.destination][0] - "timeZone" - "type" - "source"
%%     }
%%   \end{verbatim}
%% \end{enumerate}

%%
%%
%%
\section{Reordering fields}

%%%%%%
\subsection{Create a new flow}
\begin{enumerate}
\item Create a new flow named \ttt{mod4-reorder-object}
\item Drop a DW to the process area of the flow
\item Define the payload input metadata to \ttt{flight\_json}
\item Edit the sample data
\item Turn on the preview
\item Change the output to \ttt{application/dw}
\end{enumerate}

%%%%%%
\subsection{The \ttt{pluck} function}
\begin{enumerate}[resume*]
\item Use \ttt{pluck} to retrieve the values of an object
  \begin{verbatim}
    %dw 2.0
    output application/dw
    ---
    payload pluck (v,k,i) -> v
  \end{verbatim}
  \emph{
    The documentation for \ttt{pluck} is found \href{https://docs.mulesoft.com/mule-runtime/4.3/dw-core-functions-pluck}{here}.  \ttt{pluck} is very similar to \ttt{mapObject}, we still iterate an object with it but instead of returning another object we are returning an array.
  }
\item Modify the expression to retrieve the keys of an object
  \begin{verbatim}
    %dw 2.0
    output application/dw
    ---
    payload pluck (v,k,i) -> k
  \end{verbatim}
\item Modify the expression to retrieve the indexes
  \begin{verbatim}
    %dw 2.0
    output application/dw
    ---
    payload pluck (v,k,i) -> i
  \end{verbatim}
\item Repeat the last three steps but this time make use of the \ttt{\$}, \ttt{\$\$}, and \ttt{\$\$\$} placeholders
  \begin{verbatim}
    %dw 2.0
    output application/dw
    ---
    payload pluck $
  \end{verbatim}
  \emph{
    The manual just displays one of the expressions, we can form the other two on our own.
    \newline
    Can you notice the correlation between the number of \ttt{\$} and their position in the \les{}?  The same correlation is present in all function that support such placeholders.
  }
\end{enumerate}

%%%%%%
\subsection{The \ttt{reduce} function}
\begin{enumerate}[resume*]
\item Explore a \ttt{reduce} function
  \begin{verbatim}
    %dw 2.0
    output application/dw
    ---
    [3,1,2] reduce $$+$
  \end{verbatim}
\item Explore the result
  \newline
  \emph{
    The result of this expression is just a summation of all numbers in the collection
    \newline
    The \href{https://docs.mulesoft.com/mule-runtime/4.3/dw-core-functions-reduce}{documentation page} for \ttt{reduce} does a pretty good job in explaining all the details.
    \newline
    \ttt{reduce} falls under a family of functions generally referred to as \href{https://en.wikipedia.org/wiki/Fold_(higher-order_function)}{fold functions}.  Unlike these fold functions which take any data as an input and result into any data as an output \ttt{reduce} in DW accepts only arrays as inputs but it can generate any type of data as an output.
  }
\item Stop using the \ttt{\$} and \ttt{\$\$} placeholders
  \begin{verbatim}
    %dw 2.0
    output application/dw
    ---
    [3,1,2] reduce (e, acc) -> acc + e    
  \end{verbatim}
\item Provide an initial value for the accumulator
  \begin{verbatim}
    %dw 2.0
    output application/dw
    ---
    [3,1,2] reduce (e, acc=0) -> acc + e
  \end{verbatim}
  \emph{
    This initial value on the accumulator is critical because now \ttt{reduce} knows what type of data to generate.  We can replace this initial value to the empty array and inspect how the result changes from a number to the an array containing the exact same elements as the original array.  This is because the \ttt{+} operator is overloaded to operated against numbers and arrays.
  }
\item Trace the reduce function
  \begin{itemize}
  \item $1^{st}$ iteration: \ttt{(3,0) -> 0 + 3}
  \item $2^{nd}$ iteration: \ttt{(1,3) -> 3 + 1}
  \item $3^{rd}$ iteration: \ttt{(2,4) -> 2 + 4}
  \end{itemize}
  \emph{
    The secret in understanding \ttt{reduce} is understanding that the result of the previous iteration is used to set the accumulator for the next!  The result of \ttt{reduce} is the result of the final iteration.
    \newline
    There are plenty of use-cases where \ttt{reduce} is ideal!  You may as well just check the \href{https://stackoverflow.com/questions/tagged/dataweave}{stackoverflow section for dataweave}.  Finally, we shall complete the reordering of an object using \ttt{reduce} because it will make for an optimal implementation vs when using \ttt{map}. 
  }
\end{enumerate}

%%%%%%
\subsection{Reorder the fields}
\begin{enumerate}[resume*]
\item \ttt{reorder} function skeleton
  \begin{verbatim}
    %dw 2.0
    output application/dw
    var reorder = (o, ris) -> o 
    ---
    payload reorder [8,2,1,7,3,4,7,5,6,0]
  \end{verbatim}
  \emph{
    The \ttt{reorder} function takes two arguments:
    \begin{itemize}
    \item \ttt{o}: The object to reorder
    \item \ttt{ris}: An array of numbers containing the indexes of the new ordering for the object--that is the field in current position 8 will appear in position 0 of the re-ordered object, etc.
    \end{itemize}
  }
\item Reorder using \ttt{map}
  \begin{verbatim}
    %dw 2.0
    output application/dw
    var reorder = (o, ris) -> do {
      var fs = o pluck $$
      ---
      {(ris map {
          (fs[$]): o[$]
        })}
    }
    ---
    payload reorder [8,2,1,7,3,4,7,5,6,0]
  \end{verbatim}
  \emph{
    We made use of \ttt{map} to iterate the \ttt{ris} and the fact the objects can be accessed by indexes to perform the re-ordering.  Finally, destroyed the array and collapsed the objects in it using the \href{https://docs.mulesoft.com/mule-runtime/4.3/dataweave-types\#dynamic_elements}{Dynamic elements} feature.
  }
\item Rename \ttt{reorder} to \ttt{reorderMap}
\item Reorder using \ttt{reduce}
  \begin{verbatim}
    var reorder = (o, ris) -> do {
      var fs = o pluck $$
      ---
      ris reduce (e,acc={}) -> acc ++ {
        (fs[e]): o[e] 
      }
    }
  \end{verbatim}
  \emph{
    This solution is superior because there is no need to destroy arrays, collapse objects.  We obtain the final result through a single iteration as compared to two from the previous solution.
    \newline
    Always consider what type of data we need to generate out of a collection--if it is anything but an array then \ttt{reduce} should be the way to go.
  }
\end{enumerate}

%%%%%%
\subsection{Integrate field re-ordering to our main use-case}
\begin{enumerate}[resume*]
\item Copy-n-paste the \ttt{reorder} function to \ttt{mod4-flights-airports}
  \begin{verbatim}
    var reorder = (o, ris) -> do {
      var fs = o pluck $$
      ---
      ris reduce (e,acc={}) -> acc ++ {
        (fs[e]): o[e] 
      }
    }
    ---
    payload map (
    $ mapObject (v,k,i) -> {(fs2rn[k] default k): v}
    )
    map {
      ($),
      airport: airports[$.destination]
    }
  \end{verbatim}
\item Re-order every single flight object in reverse
  \begin{verbatim}
    var reorder = (o, ris) -> do {
      var fs = o pluck $$
      ---
      ris reduce (e,acc={}) -> acc ++ {
        (fs[e]): o[e] 
      }
    }
    ---
    payload map (
    $ mapObject (v,k,i) -> {(fs2rn[k] default k): v}
    )
    map {
      ($),
      airport: airports[$.destination]
    }
    map (
      $ reorder (8 to 0)
    )
  \end{verbatim}
\end{enumerate}
