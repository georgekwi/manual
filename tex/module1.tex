\chapter{Fundamentals -- Review++}
%%
%%
%%
\section{Import a basic Mule project into Anypoint Studio}
%%%%%%
\subsection{Import the starter project}
\begin{enumerate}
\item Start Anypoint Studio
\item Create a new workspace
\item Import the \texttt{apdw2-flights-starter.jar} project under the \texttt{studentFiles/mod01}
\end{enumerate}
%%%%%%
\subsection{Create new project}
\begin{enumerate}[resume*]
\item Create a new project
  \newline
  \emph{
    Creating a new project and copying only the files you minimally need for the class helps in containing the ``noise'' that is introduced with starter project.  Additionally, there is the extra benefit of not having to deal with students who are having compilation issues with the starter project.
  }
\item Create a new project and call it dataweave
\item From the apdw2-flights-starter copy the following files over to the new project:
  \begin{enumerate}
  \item \texttt{src/main/resources/airportInfoTiny.csv} to \texttt{src/main/resources}
  \item \texttt{src/main/resources/examples/mockdata/deltaSoapResponsesToAllDestinations.xml} to \texttt{src/test/resources}
  \item \texttt{src/test/resources/flight-example.json} to \texttt{src/test/resources}
  \end{enumerate}
\end{enumerate}

%%
%%
%%
\section{Fundamentals review++}
\emph{
  In this WT the goal is to attempt (I am saying attempt because often enough we have participants who don’t meet the prerequisites) to bring everyone at the same level by (1) reviewing fundamentals and (2) illustrating features of DW that we will be using throughout the class
}

%%%%%%
\subsection{Create the flow, set the metadata}
\begin{enumerate}
\item Rename the \texttt{dataweave.xml} to \texttt{mod1.xml}
\item Create a new flow named \texttt{mod1-review++}
  \newline
  \emph{
    The reason for prefixing the flow name with the name of the flow is a best-practice one.
    Such a convention will improve the readability of your flows by identifying the
    Mule Configuration file a flow is defined under by just looking at a Flow Reference's
    display name.
  }
\item Drop a DW (aka \texttt{Transform Message}) to the process area of the flow
\item Define the payload input metadata to the \texttt{src/test/resources/flight-example.json},
  set the name of the type to \texttt{flight\_json}
\item Edit the sample data
\item Turn on the preview
\item Change the output to JSON
\end{enumerate}

%%%%%%
\subsection{Construction}
\begin{enumerate}[resume*]
\item What are the semantics of \texttt{\{\}} in DW?
  \begin{enumerate}
  \item Object creation
  \end{enumerate}
\item What are the semantics of \texttt{[]} in DW?
  \begin{enumerate}
  \item Array creation
  \end{enumerate}
\end{enumerate}

%%%%%%
\subsection{Fields}
\begin{enumerate}[resume*]
\item Three different ways of accessing the field \texttt{airline}
  out of the \texttt{payload}. What are they?
  \begin{enumerate}
  \item \texttt{payload.airline}
  \item \texttt{payload["airline"]}
  \item \texttt{payload[0]}
  \end{enumerate}
  \emph{Let me let you in a secret: Objects internally are represented as arrays—field access is a façade}
\item Why DW stores objects as arrays?
  \begin{enumerate}
  \item Because DW is the only language I know of that allows the creation of objects with
    duplicate field names...
    \begin{lstlisting}
      {
        a: 1,
        a: 2,
        a: 3
      }
    \end{lstlisting}
    ... and the only way I can access the second and third field is through an index access.
    But now we have more questions that need to be answered.
  \item Why would a language allow for such a feature?  That is duplicate fields within an object.
    \begin{enumerate}
    \item Because of XML, how else you expect to be able to generate XML with tags that repeat:
      \begin{lstlisting}
        %dw 2.0
        output application/xml
        ---
        "as": {
	  a: 1,
	  a: 2,
	  a: 3
        }        
      \end{lstlisting}
    \end{enumerate}
  \end{enumerate}
\end{enumerate}

%%%%%%
\subsection{String concatenation}
\begin{enumerate}[resume*]
\item Two ways to concatenate strings
  \begin{enumerate}
  \item \texttt{"The flight is operated by " ++ payload.airline}
  \item \texttt{"The flight is operated by \$\{payload.airline\}"}
  \end{enumerate}
\item You have to be careful that the expression inside the \texttt{\$\{\}} returns
  a string, otherwise you will be getting type missmatch errors.
\end{enumerate}

%%%%%%
\subsection{Conditional expressions}
\begin{enumerate}[resume*]
\item \texttt{if then else} conditional
  \begin{enumerate}
  \item \texttt{if (true) 1 else 0}
  \item \texttt{if (false) 1 else 0}
  \end{enumerate}
\item Nullity conditional
  \begin{enumerate}
  \item lstinline{null default "Other value"}
  \item lstinline{"The value" default "Other value"}
  \end{enumerate}
\item Conditional elements
  \begin{enumerate}
  \item Objects
    \begin{lstlisting}
      {
        a: 1,
        (b: 2) if (true),
        (c: 3) if (false)
      }
    \end{lstlisting}
  \item Arrays
    \begin{lstlisting}
      [
        1, 
        (2) if (true),
        (3) if (false)
      ]
    \end{lstlisting}
  \end{enumerate}
\end{enumerate}
%%%%%%
\subsection{Array access and Ranges}
\begin{enumerate}[resume*]
\item Array access
  \begin{enumerate}
  \item \texttt{[2,6,4,1,7][0]} evaluates to \texttt{2}
  \item \texttt{[2,6,4,1,7][-1]} evaluates to \texttt{7}
  \end{enumerate}
\item Ranges
  \begin{enumerate}
  \item \texttt{0 to 5} evaluates to the \texttt{[0,1,2,3,4,5]} array 
  \item \texttt{5 to 0} evaluates to the \texttt{[5,4,3,2,1,0]} array
  \end{enumerate}
\item Ranges, Arrays, and Strings
  \begin{enumerate}
  \item \texttt{[2,6,4,1,7][1 to -2]} evaluates to the \texttt{[6,4,1,7]} sub-array
  \item \texttt{[2,6,4,1,7][-1 to 0} reverses the array
  \item \texttt{payload.airline[-3 to -1]} evaluates to the last characters in the string
  \item \texttt{payload.airline[-1 to 0]} reverses the string
  \end{enumerate}
\end{enumerate}

%%%%%%
\subsection{Common functions}
\begin{enumerate}[resume*]
\item 
\end{enumerate}

%%%%%%
\subsection{Expression chaining}

%%%%%%
\subsection{Transform XML to JSON}

%%%%%%
\subsection{Transform JSON to XML}

