\chapter{Defensive programming}

%%
%%
%%
\section{Function Overloading}

%%%%%%
\subsection{Create a new flow}
\begin{enumerate}
\item Create a new Mule Configuration file and name it \ttt{mod3}
\item Create a new flow named \ttt{mod3-data-matcher}
\item Drop a DW to the process area of the flow
\item Turn on the preview
\item Switch to the Source Only view
\item Change the output to \ttt{application/dw}
\end{enumerate}

%%%%%%
\subsection{Nullity and function signatures}
\begin{enumerate}[resume*]
\item Explore nullity
  \begin{verbatim}
    %dw 2.0
    output application/dw
    ---
    typeOf(null)
  \end{verbatim}
  \emph{
    By looking at the preview we can quickly realize that DW's typing system separates nullity from all other types.  The benefit of such separation is that you can catch a good chunk of nullity errors at compile time. 
  }
\item Replace the expression with just \ttt{sizeOf}
\item Examine the signature in the preview
  \begin{verbatim}
    (arg0:Array | Object | Binary | String) -> ???
  \end{verbatim}
  \emph{
    \ttt{sizeOf} expects a single argument.  This argument can be one of listed types separated by a pipe symbol (\ttt{|}).  This \ttt{|} indicates that this function has been overloaded.
  }
  \newline
  \emph{
    Only through \ttt{application/dw} can we see the signature of functions. 
  }
\item Replace the expression with just \ttt{map}
\item Examine the signature in the preview
  \begin{verbatim}
    (arg0:Array | Null, arg1:Function | Function) -> ???
  \end{verbatim}
  \emph{
    \ttt{map} is also an overloaded function, but this time it is a function that takes two arguments.  If we were to reconstruct these overloaded functions defining \ttt{map} then we can tell (1) that there are two overloaded functions because we have a single pipe separating the types of the arguments and (2) the arguments of these overloaded functions will look as follows by using the position of the corresponding arguments:
    \begin{itemize}
    \item \lstinline{map(Array,Function)} -- i.e. \ttt{Array} will match with the first \ttt{Function}
    \item \lstinline{map(Null, Function)} -- i.e. \ttt{Null} will match with the second \ttt{Function}
    \end{itemize}
  }
\end{enumerate}

%%%%%%
\subsection{Overload a function}
\begin{enumerate}[resume*]
\item Create a function that accepts the \ttt{Null} type
  \begin{verbatim}
    %dw 2.0
    output application/dw
    fun dataMatcher(n: Null) = "Null found"
    ---
    dataMatcher(null)
  \end{verbatim}
  \emph{
    The \ttt{:} separates the name of the argument with the type of the argument.  The same type of syntax can be used to specify types when needed.
    \newline
    You have created a function that takes only null values.
  }
\item Pass an empty array instead
  \begin{verbatim}
    %dw 2.0
    output application/dw
    fun dataMatcher(n: Null) = "Null found"
    ---
    dataMatcher([])
  \end{verbatim}
  \emph{
    You are now getting errors because our function is not defined to accept arrays.
  }
\item Overload the function to also accept arrays
  \begin{verbatim}
    %dw 2.0
    output application/dw
    fun dataMatcher(n: Null) = "Null found"
    fun dataMatcher(a: Array) = "Array found"
    ---
    dataMatcher([])
  \end{verbatim}
  \emph{
    Only a function declared through \lstinline{fun} can be overloaded--in other words you cannot overload \lep
  }
\item Display the signature of the \ttt{dataMatcher} function
  \begin{verbatim}
    %dw 2.0
    output application/dw
    fun dataMatcher(n: Null) = "Null found"
    fun dataMatcher(a: Array) = "Array found"
    ---
    dataMatcher
  \end{verbatim}
\item Examine the signature
  \begin{verbatim}
    (arg0:Null | Array) -> ???
  \end{verbatim}
  \emph{
    The signature clearly show the types the function accepts.
  }
\item Overload the function for \lstinline{Object}, \lstinline{Number}, and \lstinline{String}
  \begin{verbatim}
    %dw 2.0
    output application/dw
    fun dataMatcher(n: Null) = "Null found"
    fun dataMatcher(a: Array) = "Array found"
    fun dataMatcher(o: Object) = "Object found"
    fun dataMatcher(n: Number) = "Number found"
    fun dataMatcher(s: String) = "String found"
    ---
    dataMatcher
  \end{verbatim}
\item Examine the signature one last time
  \begin{verbatim}
    (arg0:Null | Array | Object | Number | String) -> ???
  \end{verbatim}
\end{enumerate}

%%
%%
%%
\section{Pattern Matching}

%%%%%%
\subsection{Matching literals}
\begin{enumerate}
\item Stay under the \ttt{mod3-data-matcher} flow
\item Delete the current expression
\item Use the \ttt{match} operator to match \ttt{Null} values
  \begin{verbatim}
    null match {
      case null -> "Null found"
    }
  \end{verbatim}
  \emph{
    This is the \ttt{match} operator, not to be confused with the \ttt{match} function.
  }
\item Match the number 10
  \begin{verbatim}
    10 match {
      case null -> "Null found"
      case 10 -> "Ten"
    }
  \end{verbatim}
  \emph{
    Cases are evaluated in order of appearance, ensure more general cases are below the more specific ones.
  }
\item Match the string ``ABC''
  \begin{verbatim}
    "ABC" match {
      case null -> "Null found"
      case 10 -> "Ten"
      case "ABC" -> "Alphabet"
    }
  \end{verbatim}
\item Match the empty array
  \begin{verbatim}
    [] match {
      case null -> "Null found"
      case 10 -> "Ten"
      case "ABC" -> "Alphabet"
      case [] -> "Empty array found"
    }
  \end{verbatim}
\item Match the empty object
  \begin{verbatim}
    {} match {
      case null -> "Null found"
      case 10 -> "Ten"
      case "ABC" -> "Alphabet"
      case [] -> "Empty array found"
      case {} -> "Empty object found"
    }
  \end{verbatim}
\item Add elements to the object
  \begin{verbatim}
    {a: 1} match {
      case null -> "Null found"
      case 10 -> "Ten"
      case "ABC" -> "Alphabet"
      case [] -> "Empty array found"
      case {} -> "Empty object found"
    }
  \end{verbatim}
  \emph{
    We are getting errors because there is no case matching this expression.
  }
  \newline
  \emph{
    There is no way to match inner data of objects and arrays directly in a case--i.e. adding inner data at the case level will only result in syntax errors.
  }
\item Add the default case for when there is no case matching your data
  \begin{verbatim}
    {a: 1} match {
      case null -> "Null found"
      case 10 -> "Ten"
      case "ABC" -> "Alphabet"
      case [] -> "Empty array found"
      case {} -> "Empty object found"
      else -> {message: "No match", data: $}
    }
  \end{verbatim}
  \emph{
    In the body of the \ttt{else} case we are back-referencing our data with a single \ttt{\$}.  We can back-reference our data in all other cases in the same way.  Further along we shall see how to define our own placeholder to back-reference our data so that we don't depend on a single \ttt{\$}
  }
\end{enumerate}

%%%%%%
\subsection{Matching types}
\emph{
  Lets use pattern matching to replicate function overloading from before.
}
\begin{enumerate}[resume*]
\item Remove or comment-out the expression
\item Match the \ttt{Null} type
  \begin{verbatim}
    null match {
      case is Null -> "Null found"
      else -> $
    }
  \end{verbatim}
  \emph{
    The \ttt{is} operator is used in the first case to match all values of type \lstinline{Null}.
  }
  \newline
  \emph{
    We pro-actively set the default case to just evaluate back to the expression we are matching.
  }
\item Match \lstinline{Number}, \lstinline{String}, \lstinline{Object}, and \lstinline{Array} types
  \begin{verbatim}
    [] match {
      case is Null -> {message: "Null found", data: $}
      case is Number -> {message: "Number found", data: $}
      case is String -> {message: "String found", data: $}
      case is Object -> {message: "Object found", data: $}
      case is Array -> {message: "Array found", data: $}
      else -> {message: "No match", data: $}
    }
  \end{verbatim}
  \emph{
    We are back-referencing the values we are matching using the \ttt{\$} for all our cases.
  }
\end{enumerate}

%%%%%%
\subsection{Matching by any condition}
\begin{enumerate}[resume*]
\item Replace the \ttt{\$} with a placeholder of our own choosing
  \begin{verbatim}
    [1] match {
      case n if (n is Null) -> {message: "Null found", data: n}
      case n if (n is Number) -> {message: "Number found", data: n}
      case s if (s is String) -> {message: "String found", data: s}
      case o if (o is Object) -> {message: "Object found", data: o}
      case a if (a is Array) -> {message: "Array found", data: a}
      else other -> {message: "No match", data: other}
    }
  \end{verbatim}
  \emph{
    These new placeholders are arbitrary, they can be anything of your choosing, from single letter to words--very similar to declaring variables.
  }
  \newline
  \emph{
    Using this new syntax to define our placeholders and conditionals opens an unlimited set of Boolean expressions that could be defined on a per case basis.
  }
\item Use pattern matching to distinguish the empty array, the array that has less than or equal to 100 elements, and the array that has larger than 100 elements
  \begin{verbatim}
    (0 to 200) match {
      case [] -> "Empty Array" 
      case a if (a is Array and sizeOf(a) <= 100) -> "Non-empty array with less or equal to 100 elements"
      else a -> "Non-empty array with larger than 100 elements"
    }
  \end{verbatim}
  \emph{
    For this use case we combine all of the above:
    \begin{itemize}
    \item Literal pattern matching for the empty array
    \item Conditional pattern matching on the type and size
    \item The default case for all other arrays
    \end{itemize}
  }
\end{enumerate}


%%
%%
%%
\section{Error Handling}

%%%%%%
\subsection{Create a new flow}
\begin{enumerate}
\item Create a new Mule Configuration file and name it \ttt{mod3}
\item Create a new flow named \ttt{mod3-errors}
\item Drop a DW to the process area of the flow
\item Turn on the preview
\item Switch to the Source Only view
\item Change the output to \ttt{application/dw}
\end{enumerate}

%%%%%%
\subsection{The error 10 / 0}
\begin{enumerate}[resume*]
\item Introduce an error
  \begin{verbatim}
    %dw 2.0
    output application/dw
    ---
    10 / 0
  \end{verbatim}
  \emph{
    When the evaluation of DW expressions result in errors the whole expression errors out--there are no partial results
  }
  \newline
  \emph{
    Imagine scenarios where we are working with sequences or records where some of these records our malformed, DW expressions will not provide you with partial results, instead the full expression will error out.  Error handling will allow for such partial results.
  }
\end{enumerate}

%%%%%%
\subsection{The \ttt{dw::Runtime::try} function}
\begin{enumerate}[resume*]
\item Navigate to the \href{https://docs.mulesoft.com/mule-runtime/4.3/dataweave}{Dataweave documentation}
\item Examine the signature of the \href{https://docs.mulesoft.com/mule-runtime/4.3/dw-runtime-functions-try}{\lstinline{dw::Runtime::try}} function found in the \href{https://docs.mulesoft.com/mule-runtime/4.3/dw-functions}{Dataweave Reference} section
  \newline
  \emph{
    The \ttt{dw::Runtime::try} function takes a single argument.  The argument is a \les{} that takes no arguments itself.  The body of this \les{} is the expression that we want to evaluate for errors.
  }
  \newline
  \emph{
    This \les{} is also referenced as a delegate function because through it we are delegating the execution of an expression to \ttt{dw::Runtime::try} function.  The latter will determine whether our expression has errors or not.
  }
  \newline
  \emph{
    The return value is a \ttt{TryResult<T>} structure.  Thankfully, we have \href{https://docs.mulesoft.com/mule-runtime/4.3/dw-runtime-types}{documentation} that describes this structure.
  }
\item Test the \ttt{10 / 0} expression
  \begin{verbatim}
    %dw 2.0
    output application/dw
    ---
    dw::Runtime::try(() -> 10 / 0)
  \end{verbatim}
\item Examine the result
  \begin{verbatim}
    {
      success: false,
      error: {
        kind: "DivisionByZeroException",
        message: "Division by zero",
        location: "\n4| dw::Runtime::try(() -> 10 / 0)\n                          ^^^^^^",
        stack: [
          "main (anonymous:4:24)"
        ]
      }
    }
  \end{verbatim}
  \emph{
    We no longer have errors preventing us from getting results in the preview.  Instead we are looking at a \ttt{TryResult} structure where the \ttt{success} field is set to \ttt{false} indicating that our expression has errors.  Finally the error's description can be seen in the \ttt{error} field.
  }
\item Test the \ttt{10 / 2} expression
  \begin{verbatim}
    %dw 2.0
    output application/dw
    ---
    dw::Runtime::try(() -> 10 / 2)
  \end{verbatim}
\item Examine the result
  \begin{verbatim}
    {
      success: true,
      result: 5.0
    }
  \end{verbatim}
  \emph{
    Similarly we have \ttt{success} set to \ttt{true} but instead of \ttt{error} we have the \ttt{result} field set to the value our expression evaluates to.
  }

\end{enumerate}
  
%%%%%%
\subsection{The \ttt{guard} function}
\begin{enumerate}[resume*]
\item Chain the \ttt{match} operator
  \begin{verbatim}
    %dw 2.0
    output application/dw
    ---
    dw::Runtime::try(() -> 10 / 0) match {
      case tr if (tr.success) -> tr.result
      else tr -> tr.error.message
    }
  \end{verbatim}
  \emph{
    When \ttt{success} is set to \ttt{true} just return the \ttt{result} otherwise arbitrarily just return the \ttt{error.message} 
  }
\item Extrapolate to a function that guards against errors
  \begin{verbatim}
    var guard = (fn) -> dw::Runtime::try( fn ) match {
      case tr if (tr.success) -> tr.result
      else tr -> tr.error.message
    }
  \end{verbatim}
  \emph{
    There is no need for a \ttt{guard} function.  There are additional error handling functions that we shall see further along the material that provide for complete and flexible error handling.  Nonetheless, going through the refactoring of our code into a function allows us to take a peek on functional programming.
  }
\item Test the \ttt{guard} function with a correct expression
  \begin{verbatim}
    guard( () -> 10 / 2 )
  \end{verbatim}
\item Test the \ttt{guard} function with an erroneous expression
  \begin{verbatim}
    guard( () -> 10 / 0 )
  \end{verbatim}  
\end{enumerate}


%%
%%
%%
\section{Optional: Partial Results}
\emph{
  Let’s see how we can get partial results now that we know how to capture errors.
  How many times you had your DW code fail because one or very few records were malformed?
  We will replicate such a scenario by creating an array containing the string representation of dates and then casting them into a date type. One of our dates will be malformed
}

%%%%%%
\subsection{List of String Dates}
\begin{enumerate}
\item Stay in the \ttt{mod3-errors} flow
\item Create an array of string dates and assigned it in a variable
  \begin{verbatim}
    var dates = [
      "2020/01/01",
      "20100101",
      "2000/01/01"
    ]
  \end{verbatim}
  \emph{
    It is the second date that is malformed and it is the cause of failure of our transformation.
  }
\item Remove or comment out the current expression
\item Iterate over dates and cast into a \ttt{Date} type
  \begin{verbatim}
    dates map (
      $ as Date {format: "yyyy/MM/dd"}
    )
  \end{verbatim}
  \emph{
    There is an error thrown telling us that the second-string date cannot be casted.
  }
\end{enumerate}

%%%%%%
\subsection{Using \ttt{guard} and obtaining partial results}
\begin{enumerate}[resume*]
\item Delegate the execution of the type-casting expression to the \ttt{guard} function
  \begin{verbatim}
    dates map (
      guard( () -> $ as Date {format: "yyyy/MM/dd"} )
    ) 
  \end{verbatim}
\item Finally \ttt{filter} for \ttt{Date} types and get your partial results while ignoring the rest
  \begin{verbatim}
    dates map (
      guard( () -> $ as Date {format: "yyyy/MM/dd"} )
    )
    filter ($ is Date)
  \end{verbatim}
\item We can expand on this solution such that we also obtain the errors
  \begin{verbatim}
    do {
      var parsedDates = dates map (
      guard( () -> $ as Date {format: "yyyy/MM/dd"} )
      )
      ---
      {
        success: parsedDates filter ($ is Date),
        failure: parsedDates filter  not ($ is Date)
      }
    }
  \end{verbatim}
  \emph{
    Note to instructors: This is a good exercise to have students work on their own for ``bonus points'', after all we have already covered all topics they need to know to get it done.
  }
\end{enumerate}
